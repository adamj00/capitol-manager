% Opcje klasy 'iithesis' opisane sa w komentarzach w pliku klasy. Za ich pomoca
% ustawia sie przede wszystkim jezyk oraz rodzaj (lic/inz/mgr) pracy.
\documentclass[shortabstract]{iithesis}
\usepackage{amssymb}
%\usepackage[utf8]{inputenc}

%%%%% DANE DO STRONY TYTUŁOWEJ
% Niezaleznie od jezyka pracy wybranego w opcjach klasy, tytul i streszczenie
% pracy nalezy podac zarowno w jezyku polskim, jak i angielskim.
% Pamietaj o madrym (zgodnym z logicznym rozbiorem zdania oraz estetyka) recznym
% zlamaniu wierszy w temacie pracy, zwlaszcza tego w jezyku pracy. Uzyj do tego
% polecenia \fmlinebreak.
\polishtitle    {Aplikacja webowa wspomagająca zarządzanie\fmlinebreak pracownikami obsługi widowni teatru}
\englishtitle   {Web Application Supporting the Management of Theater Audience Service Staff}
\polishabstract {Celem pracy jest stworzenie aplikacji webowej wspomagającej zarządzanie pracownikami obsługi widowni teatru umożliwiającej układanie i przeglądanie grafiku, zaznaczanie dyspozycyjności, konfigurację stanowisk oraz raportowanie czasu pracowników wraz z możliwością wyświetlania podsumowań miesięcznych. Wszystko w jednej aplikacji dostępnej z poziomu przeglądarki. Wymagania funkcjonalne i niefunkcjonalne zostały sporządzone na podstawie doświadczeń autora w pracy w Teatrze Muzycznym Capitol we Wrocławiu. Docelowo aplikacja ta może zastąpić rozwiązania dotychczas używane w tym teatrze przez pracowników obsługi widowni.}
\englishabstract{\ldots}
% w pracach wielu autorow nazwiska mozna oddzielic poleceniem \and
\author         {Adam Jarząbek}
% w przypadku kilku promotorow, lub koniecznosci podania ich afiliacji, linie
% w ponizszym poleceniu mozna zlamac poleceniem \fmlinebreak
\advisor        {dr inż. Leszek Grocholski}
\date          {20 czerwca 2024}                     % Data zlozenia pracy
% Dane do oswiadczenia o autorskim wykonaniu
%\transcriptnum {}                     % Numer indeksu
%\advisorgen    {dr. Jana Kowalskiego} % Nazwisko promotora w dopelniaczu
%%%%%

%%%%% WLASNE DODATKOWE PAKIETY
%
\usepackage{graphicx,listings,amsmath,amssymb,amsthm,amsfonts,tikz}
%
%%%%% WŁASNE DEFINICJE I POLECENIA
%
%\theoremstyle{definition} \newtheorem{definition}{Definition}[chapter]
%\theoremstyle{remark} \newtheorem{remark}[definition]{Observation}
%\theoremstyle{plain} \newtheorem{theorem}[definition]{Theorem}
%\theoremstyle{plain} \newtheorem{lemma}[definition]{Lemma}
%\renewcommand \qedsymbol {\ensuremath{\square}}
% ...
%%%%%

\begin{document}

%%%%% POCZĄTEK ZASADNICZEGO TEKSTU PRACY

\chapter{Wprowadzenie}

\section{Opis problemu}

Teatry jako instytucje kultury odgrywają kluczową rolę w życiu społecznym, oferując widzom różnorodne formy rozrywki i edukacji. Jednak za kulisami, efektywne funkcjonowanie takiego miejsca wymaga skomplikowanej organizacji pracy wielu osób. W Teatrze Muzycznym Capitol, gdzie pracuję jako pracownik obsługi widowni, codzienne wyzwania związane z koordynacją pracy około 40 bileterów stanowią istotny problem operacyjny. Bileterzy pełnią różne funkcje – od sprawdzania biletów, przez obsługę szatni, aż po pilnowanie porządku w trakcie spektakli. Zarządzanie ich pracą, ze względu na zmienną dyspozycyjność i różnorodność stanowisk, wymaga elastycznego podejścia i precyzyjnego planowania.

Do tej pory proces ten był realizowany za pomocą arkusza kalkulacyjnego Google Sheets oraz tradycyjnej papierowej dokumentacji. Taki sposób zarządzania, choć częściowo efektywny, posiada liczne ograniczenia. Brak centralizacji danych, trudności w szybkim dostępie do aktualnych informacji, ograniczone możliwości personalizacji oraz brak automatyzacji w przydzielaniu stanowisk to tylko niektóre z problemów, które zidentyfikowałem jako pracownik i obserwator.

W niniejszej pracy inżynierskiej postawiłem sobie za cel stworzenie aplikacji webowej, która usprawni proces zarządzania pracownikami obsługi widowni w Teatrze Muzycznym Capitol. Aplikacja ta ma umożliwić pracownikom łatwe i intuicyjne zgłaszanie swojej dyspozycyjności, przeglądanie przydzielonych im wydarzeń i stanowisk oraz dostęp do kluczowych informacji o spektaklach i scenach. Z kolei dla kierownictwa teatru aplikacja ma oferować narzędzia do efektywnego zarządzania zasobami ludzkimi, w tym konfigurację scen, rodzajów stanowisk, tworzenie wydarzeń oraz automatyczne przydzielanie pracowników do zadań.

Zakres pracy obejmuje analizę istniejącego problemu, projektowanie rozwiązania, implementację systemu oraz jego testowanie i wdrożenie. W dalszych rozdziałach zostaną przedstawione szczegółowe informacje dotyczące każdego z tych etapów, wraz z oceną wpływu wprowadzonych zmian na poprawę funkcjonowania obsługi widowni w Teatrze Muzycznym Capitol.

\section{Przegląd istniejących rozwiązań}

Przed opisem proponowanego rozwiązania warto przyjrzeć wadom i zaletom rozwiązania aktualnie używanego w teatrze oraz alternawnym rozwiązaniom komercyjnym.

\subsection{Arkusze Google Sheets}

\subsubsection{Opis rozwiązania}
Arkusze Google Sheets stanowią obecnie podstawowe narzędzie do zarządzania pracownikami obsługi widowni w Teatrze Muzycznym Capitol. Cały system oparty na arkuszach kalkulacyjnych składa się z kilku zintegrowanych zakładek, z których każda pełni określoną funkcję w procesie planowania pracy pracowników.

Pierwszą z nich jest zakładka ,,dyspozycyjność'', która ma postać tabeli z kolumną zawierającą imiona i nazwiska pracowników po lewej stronie. Każda kolejna kolumna odpowiada poszczególnym wydarzeniom, zawierając informacje takie jak data, nazwa spektaklu, godzina oraz scena. Pracownicy zaznaczają w tej tabeli swoją dyspozycyjność na dane wydarzenie, wpisując ,,tak'' w odpowiedniej komórce. Brak dyspozycyjności oznaczany jest poprzez wpisanie ,,nie''. Po upływie terminu na zaznaczanie dyspozycyjności, kierownik ukrywa zakładkę ,,dyspozycyjność'' i tworzy nową zakładkę ,,zatwierdzone''.


\begin{figure}[h] % Opcja 'h' oznacza 'here', czyli umieszczenie obrazka w miejscu definicji
\centering % Wycentrowanie obrazka
\includegraphics[width=1.0\textwidth]{dyspo.png} % Wstawienie obrazka
\caption{Fragment akrusza do zaznaczania dyspozycyjności.} % Podpis pod obrazkiem
\end{figure}


Arkusz ,,zatwierdzone'' ma podobny układ do arkusza ,,dyspozycyjność'', ale słowo ,,tak'' przy danym wydarzeniu i pracowniku oznacza, że pracownik został oficjalnie przydzielony do pracy na tym wydarzeniu. Zazwyczaj do jednego wydarzenia przypisywanych jest, w zależności od sceny, 17 lub 6 pracowników.

\begin{figure}[h] % Opcja 'h' oznacza 'here', czyli umieszczenie obrazka w miejscu definicji
\centering % Wycentrowanie obrazka
\includegraphics[width=1.0\textwidth]{zatwierdzone.png} % Wstawienie obrazka
\caption{Fragment akrusza do z ,,Zatwierdzone'' z częściowo przypisanymi stanowiskami.} % Podpis pod obrazkiem
\label{fig:dyspo} % Etykieta do odwoływania się do obrazka w tekście
\end{figure}

Dodatkowo, istnieje arkusz ,,Bileterzy'', który zawiera tabelę z imionami, nazwiskami oraz numerami telefonów pracowników i koordynatora. W miarę zbliżania się danego wydarzenia, w zakładce ,,zatwierdzone'', w komórkach z ,,tak'', dopisywane są konkretne stanowiska, na których pracownicy mają pełnić swoje obowiązki.

Raportowanie czasu pracy bileterów odbywa się poprzez fizyczną dokumentację uzupełnianą po każdym wydarzeniu.

\subsubsection{Wady i zalety}
Wady i zalety tego rozwiązania są następujące:

\textbf{Zalety:}
\begin{itemize}
  \item Dostępność – Arkusze Google Sheets są dostępne z każdego miejsca i na każdym urządzeniu z dostępem do internetu.
  \item Możliwości - dzięki wykorzystaniu Google Sheets kordynator może korzystać z dobrodziejstw arkuszy kalkulacyjnychtakich takich jak formuły i formatowanie warunkowe.
  \item Znajomość narzędzia – Wielu użytkowników jest już zaznajomionych z obsługą arkuszy kalkulacyjnych, co obniża próg wejścia dla nowych pracowników.
\end{itemize}

\textbf{Wady:}
\begin{itemize}
  \item Brak personalizacji – Wszyscy pracownicy korzystają z tego samego widoku arkusza, co oznacza brak indywidualnych tabel do zarządzania własną dyspozycyjnością czy grafikiem pracy.
  \item Problemy z bezpieczeństwem – Dostęp do arkusza mają wszystkie osoby, które posiadają link, co stwarza ryzyko nieautoryzowanego dostępu i manipulacji danymi.
  \item Brak automatyzacji – Proces przydzielania zadań jest całkowicie manualny, co zwiększa ryzyko błędów i jest czasochłonne dla kierownictwa.
  \item Trudności z aktualizacją danych – W przypadku zmian w grafiku, nie ma prostego sposobu na powiadomienie pracowników, co może prowadzić do nieporozumień i błędów.
  \item Ryzyko utraty danych – Błędne operacje użytkowników mogą prowadzić do przypadkowego usunięcia ważnych informacji bez możliwości łatwego odzyskania.
\end{itemize}

\subsection{When I Work}

\subsubsection{Opis rozwiązania}
When I Work to komercyjne oprogramowanie zaprojektowane do zarządzania grafikami pracy, które umożliwia pracownikom zgłaszanie swojej dyspozycyjności oraz umożliwia menedżerom szybkie tworzenie i modyfikowanie grafików pracy. Aplikacja oferuje możliwość komunikacji z pracownikami i posiada aplikację mobilną, co pozwala na łatwe zarządzanie zmianami w pracy z dowolnego miejsca.
\begin{figure}[h]
    \centering
    \includegraphics[width=1.0\linewidth]{wheniwork.jpg}
    \caption{Aplikacja do zarządania pracownikami When I Work}
    \label{fig:wiw}
\end{figure}

\subsubsection{Wady i zalety}

\textbf{Zalety:}
\begin{itemize}
  \item Intuicyjny interfejs użytkownika, który ułatwia zarządzanie grafikami pracy.
  \item Funkcja komunikacji w aplikacji pozwala na szybkie przekazywanie informacji między pracownikami a zarządem.
  \item Aplikacja mobilna umożliwia pracownikom łatwy dostęp do grafiku pracy i zgłaszanie dyspozycyjności w dowolnym miejscu i czasie.
  \item Automatyzacja procesu tworzenia grafiku może znacznie oszczędzić czas menedżerów i zwiększyć efektywność planowania.
\end{itemize}

\textbf{Wady:}
\begin{itemize}
  \item Koszt subskrypcji może być barierą dla mniejszych organizacji z ograniczonym budżetem.
  \item Wymagany czas na szkolenie pracowników w zakresie korzystania z nowego systemu.
  \item Zależność od dostępu do internetu i urządzeń mobilnych może być ograniczeniem w niektórych środowiskach pracy.
\end{itemize}

\section{Funkcje aplikacji}

Poniżej wymienionio wymagania funkcjonalne i niefunkcjonalne aplikacji. Zostały one sporządzone na podstawie doświadczeń autora w pracy w jednym z wrocławskich teatrów w roli biletera oraz na podstawie wywiadu zebranego wśród pracowników tej instytucji.

\subsection{Wymagania funkcjonalne}

Wymagania funkcjonalne zostały podzielone na trzy kategorie, w zależności od roli użytkownika w systemie: wymagania dla bileterów, wymagania dla koordynatorów oraz wymagania ogólne.

\subsubsection*{Wymagania dla bileterów}
\begin{enumerate}
  \item Bileter może wyświetlać informacje kontaktowe innych pracowników.
  \item Bileter może wyświetlać informacje o spektaklach oraz o scenach, na których odbywają się wydarzenia.
  \item Bileter może wyświetlać wszystkie wydarzenia z podziałem na grupy wydarzeń (np. na miesiące).
  \item Bileter może wyświetlać informacje o wydarzeniach, do których jest przypisany, wraz z informacją o stanowisku pracy.
  \item Bileter może określać swoją dyspozycyjność na dane wydarzenia.
  \item Bileter może przeglądać informacje o wydarzeniach z obecnego tygodnia, do których jest przypisany.
  \item Bileter może raportować swój czas pracy oraz wyświetlać łączny czas pracy w danym miesiącu.
\end{enumerate}

\subsubsection*{Wymagania dla koordynatorów}
\begin{enumerate}
  \item Koordynator może tworzyć, usuwać i edytować konta użytkowników.
  \item Koordynator może określać jakie istnieją stanowiska pracy biletera oraz określać ich typ.
  \item Koordynator może tworzyć i edytować sceny, na których odbywają się spektakle, w tym określać, jakie i ile stanowisk trzeba obsadzić, aby obsłużyć wydarzenie odbywające się na tej scenie.
  \item Koordynator może tworzyć i edytować spektakle grane w teatrze.
  \item Koordynator może tworzyć i usuwać wydarzenia (spektakl + data i godzina) z podziałem na grupy.
  \item Koordynator może włączać i wyłączać możliwość uzupełniania dyspozycyjności dla danej grupy wydarzeń oraz sterować widocznością danych grup wydarzeń w grafiku.
  \item Koordynator może przypisywać pracowników do wydarzeń oraz przypisywać stanowiska do pracowników.
  \item Koordynator ma możliwość wyświetlania raportów pracy bileterów.
\end{enumerate}

\subsubsection*{Wymagania ogólne}
\begin{enumerate}
  \item Użytkownik może zalogować się do systemu używając adresu e-mail oraz hasła.
  \item W zależności od przydzielonej roli, użytkownik będzie miał dostęp do odpowiednich funkcjonalności. Przewidziane role: Bileter oraz Koordynator.
  \item Aplikacja ma umożliwiać automatyczne przypisanie stanowisk do pracowników w danej grupie wydarzeń według sensownej heurystyki.
\end{enumerate}

\subsection{Wymagania niefunkcjonalne}
System powinien spełniać następujące wymagania niefunkcjonalne:
\begin{enumerate}
  \item Czas odpowiedzi systemu na żądania użytkownika nie powinien przekraczać 2 sekund.
  \item Interfejs użytkownika powinien być intuicyjny i dostosowany do potrzeb użytkowników o różnym stopniu zaawansowania.
  \item Aplikacja powinna być dostępna na najpopularniejszych systemach operacyjnych i przeglądarkach internetowych, w tym na Windows, macOS, Linux, Chrome, Firefox, Safari i Edge w ich aktualnych wersjach.
  \item Wszelkie aktualizacje systemu powinny odbywać się z minimalnym wpływem na dostępność i wydajność systemu.
  \item Dokumentacja systemu powinna być kompletna, aktualna i łatwo dostępna dla użytkowników oraz administratorów systemu.
\end{enumerate}

\chapter{Podręcznik użytkownika}

W tej części przedstawiona zostanie instrukcja obsługi aplikacji, zarówno z perspektywy biletera, jak i koordynatora. Wersja aplikacji, która zostanie udostępniona użytkownikowi, zależy od roli przypisanej do jego konta.

\section{Dostęp do aplikacji}
Aby uzyskać dostęp do aplikacji, należy wejść na stronę (tu link) i zalogować się, używając adresu e-mail i hasła. Inną metodą jest lokalne uruchomienie aplikacji za pomocą środowiska IntelliJ lub podobnego, co skutkuje użyciem lokalnej bazy danych zainicjalizowanej danymi testowymi.

\section{Instrukcja obsługi - część ogólna}

\subsection{Strona główna}

Na ekranie głównym serwisu wyświetlana jest lista wydarzeń zaplanowanych na obecny tydzień, do których użytkownik jest przypisany. Aby zobaczyć szczegóły wydarzenia, w tym przypisane stanowisko, należy kliknąć na etykietę reprezentującą dane wydarzenie.

Poniżej listy nadchodzących zmian znajduje się kalendarz z wydarzeniami przypisanymi do użytkownika w bieżącym miesiącu.

\begin{figure}[h!]
    \centering
    \includegraphics[width=1.0\linewidth]{ekran_glowny_bileter.png}
    \caption{Ekran główny serwisu}
    \label{fig:glowny}
\end{figure}

\subsection{Nawigacja}
Pasek nawigacyjny znajdujący się u góry każdego widoku umożliwia łatwe przełączanie się między różnymi sekcjami aplikacji.

\subsection{Wylogowywanie się i zmiana hasła}
Aby wylogować się z konta, należy wybrać z paska nawigacyjnego opcję \textit{Moje konto} $\rightarrow$ \textit{Wyloguj}.
W celu zmiany hasła, użytkownik powinien wybrać \textit{Moje konto} $\rightarrow$ \textit{Zmień hasło}.

\section{Instrukcja obsługi - część dla bileterów}

\subsection{Wyświetlanie informacji o pracownikach}

Informacje o pracownikach zarejestrowanych w systemie można znaleźć w sekcji \textit{Pracownicy}. Zawiera ona imiona, nazwiska oraz numery telefonów pracowników. Koordynatorzy są wyróżnieni na początku listy.

% Tu wstawić obrazek kiedy będą już dane

\subsection{Wyświetlanie informacji o scenach i spektaklach}

\subsubsection{Sceny}
\label{stages}
Sekcja \textit{Informacje} $\rightarrow$ \textit{Sceny} zawiera tabelę z informacjami o wszystkich scenach. Tabela ta obejmuje nazwę sceny, liczbę miejsc siedzących oraz adres.

% Obrazek

\subsubsection{Spektakle}
W sekcji \textit{Informacje} $\rightarrow$ \textit{Spektakle} znajdują się informacje o spektaklach wystawianych w teatrze. Dla każdego spektaklu dostępne są takie dane, jak tytuł, scena, czas trwania oraz dodatkowe informacje.

%Obrazek

\subsection{Grafiki}

W sekcji \textit{Grafik} dostępne są aktywne grupy wydarzeń. Jeżeli dla danej grupy wydarzeń istnieje grafik, można go wyświetlić, używając przycisku \textit{Pokaż wydarzenia}. Wyświetlana jest również liczba zmian przypisanych do użytkownika w tej grupie.

Jeśli w danej grupie wydarzeń możliwe jest wypełnienie dyspozycyjności, można to zrobić, klikając przycisk \textit{Wypełnij dyspozycyjność}. Na kafelku grupy wyświetlany jest również stopień wypełnienia dyspozycyjności oraz procentowa dyspozycyjność, czyli procent wydarzeń, na które użytkownik zaznaczył ,,Tak''.

\begin{figure}[h!]
    \centering
    \includegraphics[width=1\linewidth]{grupy_wydarzen.png}
    \caption{Grupy wydarzeń. Dla grupy \textit{Maj 2024} można wyświetlić wydarzenia, a dla grupy \textit{Czerwiec 2024} można wypełnić dyspozycyjność.}
    \label{fig:event-groups}
\end{figure}

\subsection{Zaznaczanie dyspozycyjności}

Po wybraniu opcji \textit{Zaznacz dyspozycyjność} pojawia się widok kalendarza, na którym można zaznaczyć swoją dostępność dla wydarzeń z wybranej grupy. Każde wydarzenie zawiera informacje o tytule spektaklu, scenie, godzinie rozpoczęcia oraz czasie trwania. Aby określić swoją dyspozycyjność dla danego wydarzenia, należy zaznaczyć opcję ,,Tak'' lub ,,Nie'' w odpowiednim miejscu. Po zaznaczeniu, wybór jest automatycznie zapisywany.

\begin{figure}[h!]
    \centering
    \includegraphics[width=1\linewidth]{dyspo.png}
    \caption{Ekran do uzupełniania dyspozycyjności. Wydarzenia oznaczone na zielono wskazują na dostępność, a te oznaczone na czerwono na jej brak.}
    \label{fig:availability}
\end{figure}

\subsection{Wyświetlanie grafiku}

Po wybraniu opcji \textit{Pokaż wydarzenia} wyświetlany jest kalendarz z wydarzeniami należącymi do danej grupy. Każde wydarzenie prezentuje tytuł spektaklu, nazwę sceny, godzinę rozpoczęcia, czas trwania oraz przypisane stanowisko (jeśli zostało przydzielone).

Wydarzenia, do których użytkownik został oficjalnie przypisany, są zaznaczone na zielono.

\begin{figure}[h!]
    \centering
    \includegraphics[width=1\linewidth]{grafik.png}
    \caption{Widok prezentujący wydarzenia z danej grupy. Przypisane wydarzenia są zaznaczone na zielono.}
    \label{fig:schedule}
\end{figure}

\subsection{Raportowanie czasu pracy}

Aby zaraportować czas pracy po zakończonej zmianie, należy wybrać opcję \textit{Raportowanie czasu pracy} z paska nawigacyjnego.

Na ekranie można wyświetlić listę raportów z wybranego miesiąca oraz sumę przepracowanych godzin. Aby wybrać miesiąc, należy wybrać odpowiednią nazwę miesiąca i rok, a następnie kliknąć przycisk \textit{Filtruj}. Aby usunąć raport, należy użyć przycisku \textit{Usuń}.

% Obrazek - lista raportów

Aby utworzyć nowy raport, należy wybrać opcję \textit{Utwórz}. Na kolejnym ekranie należy określić datę, godzinę rozpoczęcia i zakończenia zmiany (z dokładnością do 15 minut). Aby zatwierdzić, należy kliknąć przycisk \textit{Zatwierdź}.

\section{Instrukcja obsługi - część dla koordynatora}

\subsection{Lista pracowników}
\label{user-list}

Aby wyświetlić listę użytkowników należy wybrać opcję \textit{Pracownicy} z menu. Ukaże się tabela zawierająca informacje o koordynatorach i biletarach

% Obrazek

\subsection{Edycja i tworzenie pracowników}

\subsubsection{Tworzenie nowego konta}

Aby stworzyć nowe konto pracownika należy wcisnąć przycisk \textit{Utwórz} na widoku \nameref{user-list}. Następnie należy określić adres e-mail, imię, nazwisko i numer telefonu pracownika oraz przypisać jedną z dwóch ról.

Konto zostanie stworzone po wciśnięciu przycisku \textit{Zapisz}. Nowym konto przypisywane jest domyślne hasło \textit{capitol}.

% Obrazek

\subsubsection{Edycja konta}

Aby edytować dane pracownika należy wcisnąć przycisk \textit{Edytuj} przy jego danych na widoku \nameref{user-list}. Na następnym widoku można edytować dane i zapisać je przyciskiem \textit{Zapisz}.

\subsection{Konfiguracja stanowisk}

Aby edytować stanowiska, na jakich mogą pracować bileterzy, należy przejść do sekcji \textit{Konfiguracja} \rightarrow \textit{Stanowiska}. Stanowiska są przypisywane do scen

W celu stworzenia nowego stanowiska należy wybrać opcję \textit{Utwórz} i określić nazwę stanowiska oraz jego typ. Następnie należy wcisnąć przycisk \textit{Zapisz}.

W celu edycji stanowiska należy edytować jego dane, a następnie wcisnąć przycisk \textit{Zapisz}.



% Obrazek - nowy raport


%%%%% BIBLIOGRAFIA

%\begin{thebibliography}{1}
%\bibitem{example} \ldots
%\end{thebibliography}

\end{document}
